
\RequirePackage{amsmath}
\documentclass{svjour3}
\usepackage{float}
\usepackage[margin=1in]{geometry}
\usepackage[numbers, sort&compress]{natbib}
\usepackage{pdfpages}
\usepackage{rotating}
\usepackage{graphicx}
\usepackage{booktabs}
\usepackage[strings]{underscore}
\usepackage{anyfontsize}
\usepackage{subfigure}
\usepackage{lipsum}
\usepackage[utf8]{inputenc}

\begin{document}






\abstract


\section{Introduction}
  
\section{Spatial models}
The section outlines two purely spatial models that utilize only the locations of past fire in order to forecast future fire counts at the census tract level. The first is a naive ``spatial histogram'' model that serves as a performance baseline for all subsequent models described in this paper. The second employs Kernel Density Estimation (KDE), which is a statistical method commonly used to generate incident heatmaps. 

\subsection{Naive count forecasting}
The first forecasting technique described in this paper is a naive count model that can be thought of as a spatial histogram with the census tracts representing a set of non-uniform bins. The estimate for the count density rate in census tract \textit{j} covered by fire department \textit{i} is described by equation \ref{eqn:naive_count}:

\begin{equation}
  \label{eqn:naive_count}
  \hat{\lambda}_{count,i,j} = \frac{F_{train,i,j}}{n_{train}A_{i,j}} 
\end{equation}

Where $\hat{\lambda}_{count,i,j}$ is the estimated average fire density per year, $F_{train,i,j}$ is the total number of fires in census tract \textit{i, j} that occurred during the training interval, $n_{train}$ is the number of years that comprise the training interval (6 years), and $A_{i,j}$ is the land area of census tract \textit{i, j}.






\clearpage
\bibliographystyle{unsrtnat}
\bibliography{./papers/references}
\end{document}
